
\begin{abstract}

Data replication results in a fundamental trade-off between operation
latency and consistency. At the weak end of the spectrum of possible
consistency models is eventual consistency, which provides no limit to
the staleness of data returned. However, anecdotally, eventual
consistency is often ``good enough'' for practitioners given its
latency and availability benefits. In this work, we explain this
phenomenon and demonstrate that, despite their weak guarantees,
eventually consistent systems regularly return consistent data while
providing lower latency than their strongly consistent
counterparts. To quantify the behavior of eventually consistent
stores, we introduce Probabilistically Bounded Staleness (PBS), a
consistency model that provides expected bounds on data staleness with
respect to both versions and wall clock time. We derive a closed-form
solution for versioned staleness and model real-time staleness for a
large class of quorum replicated, Dynamo-style stores. Using PBS, we
measure the trade-off between latency and consistency for partial,
non-overlapping quorum systems under Internet production workloads. We
quantitatively demonstrate how and why eventually consistent systems
frequently return consistent data within tens of milliseconds while
offering large latency benefits.
%, where it currently ships as a feature of the software distribution.

\end{abstract}
